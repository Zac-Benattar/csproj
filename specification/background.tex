\section{Background}
Obtaining a private pilot's license is one of the first steps in any commercial pilot's career, and often the end goal of any prospective private pilot.
The preparation and examination for such a qualification is often a very costly endeavour, mainly due to the need for frequent practice with qualified instructors.
Many students hence have to go into their examinations not feeling confident in their abilities.
One possible solution for this is for students to use online practice systems which helps prepare them for their specific exams, alongside occasional practice with qualified instructors.
This helps reduce the high financial burden of obtaining a pilot's license and allows substantially more practice than would be available with a human instructor.
\section{Existing software}
Two online practice apps exist currently which provide students an experience close to that which they will encounter in the exam, Readability 5 \cite{readability5} and Wilco Radio \cite{wilcoradio}.
Both systems provide a simulated RTC which the student can communicate with via a text box, with Wilco Radio also recently implementing a voice recognition system.
Though very featured, a limitation of these programs is that scenarios come from a pool of predefined options, and hence students can learn the correct responses to each scenario without actually understanding the underlying concepts or may find the scenarios insufficient in quantity.

The Coronavirus pandemic helped highlight the need for refresher training for pilots, with the European Union Aviation Safety Agency (EASA) releasing guidelines on extra training for pilots and ground staff who are coming back from an extended break from work \cite{EASA-training-post-covid}.