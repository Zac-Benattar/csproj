\chapter{Conclusions}
\label{ch:conclusions}

The project is a success. Summarise what you have done and accomplished.

\section{Future work}
\label{FutureWork}

While RT-Trainer has met most of its requirements when it was designed, during the course of development it became clear that the system could have a much wider set of use cases, upon certain additions. These are listed below, and each represent a significant undertaking, but if RT-Trainer is used as a starting point, they should all be feasible to develop in a similar amount of time that the RT-Trainer project took to complete.

\subsection{Teaching}
Currently the system is simply a practice tool, despite being named RT-Trainer. In order to live up to its name some level of teaching should be included, such that users can learn R/T, then practice it using the existing system, without ever leaving the website.

\subsection{Flight planner integration and live data}
The route planner allows users to enter in their own routes, callsign and aircraft type. This allows users to practice routes that in theory they could fly in real life. An overhaul of the scenario system with similar control as the route planner has could give users the ability to properly use the system to practice a route they plan to actually fly. Live data would be required for the greatest usefulness. As NOTAMs are published online, and all other aeronautical data is available up to date online, this could be a feasible extension to the current system.

\subsection{Instrument Flight Rules (IFR) support}
The FRTOL exam is only relevant for Visual Flight Rules (VFR) flight. Once a pilot has been given a license to fly VFR, they may choose to obtain an IFR license. The RT is very similar, and could just be an addition to what is currently supported. Commercial airline and airfreight flights use IFR, and hence RT-Trainer could be part of the training for a commercial pilot learning R/T for their job.