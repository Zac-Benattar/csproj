\chapter{Background}
\label{ch:background}

The use of automatic speech recognition has been trialled in an aviation communication context by various researchers over the last decade, but so far these efforts have been related to reducing workload or improving reliability of ATC's (Air Traffic Controller) radio calls. Projects such as the EU's STARFiSH (Safety and Artificial Intelligence Speech Recognition) have been funded by a number of government agencies as modernisation efforts for ATC processes with some success \cite{STARFiSH}. The main challenge with such efforts is found in the safety aspect. This project however, is related to the other party in the radio communication, the pilot.

% https://www.malorca-project.de/wp/wp-content/uploads/Ohneiser_Oliver_1474.pdf
% Paper on using a similar system to provide suggestions/optional decisions for ATC based on their radio calls and a model of the status of each flight they are managing
%  Although the usage of data link in ATC is discussed at least since the 90s, voice communication will definitely remain a pillar of air traffic control. The Strategic Research & Innovation Agenda (SRIA) of ACARE (ACARE, 2012) or Flightpath 2050 (European Commission, 2011) do not expect a fully automated ATM (Air Traffic Management) system in the next decades.

Pilots' R/T skills have been identified as a common weak spot by many aviation agencies \cite{flight-safety-failure-to-communicate}. A study by Istanbul Rumeli University recently found that 7 of the 20 deadliest aircraft collisions were caused by communication errors \cite{communication-in-accidents}. Famously, the deadliest aviation accident in history, the Tenerife airport disaster, was caused primarily by miscommunication \cite{tenerife-accident-description}. The pilot of KLM flight 4805 mistaking an instruction from ATC referring to takeoff without giving permission, as permission to take off \cite{tenerife-accident-description}, resulting in an impact with Pan Am flight 1736 taxiing down the same runway. Significant changes to communication procedures and the grammar of R/T were implemented after the disaster \cite{CAP413-ed15-ch2-p6}, and changes continue to be implemented following other incidents.

The addition of data links to ATCs current set of tools for managing aircraft has been discussed since at least the 90s, but voice communication shows no sign of being replaced as the primary mode of communication between aircraft \cite{609472}. The introduction of a new technology to an ATC's toolkit would still require them to train for its failure, and a subsequent return to manually performing the task of the new technology, and often more radio calls. Given the fundamental nature of radio call based communication in aviation and the industry's safety requirements, correctness, brevity and confidence are important for spoken calls. Both pilots and ATCs must be correct in their wording to avoid dangerous situations. Communications are kept short as radio communications support only one speaker at a time. Confidence ensures that calls are made at low workload times, and are not the main focus of the pilot, which should be controlling the aircraft. The project's client, a R/T instructor based at Wellesbourne airfield, has encountered many pilots, both student and commercially operating, with poor R/T skills in the air.

The Coronavirus pandemic has increased the need for pilot refresher training, with the European Union Aviation Safety Agency (EASA) releasing guidelines on extra training for pilots and ground staff who are coming back from an extended break from work \cite{EASA-Training-Post-Covid}. Given the nature of the flight profession, pilots may not be able to attend regular in-person training if they are travelling. A system that offers online solo practice with no need for arranging an instructor could form part of a pilot’s continued skills practice.

% End this nicely as a way of saying that new software is needed to attempt to solve some of these background issues
